Código del archivo servidor\_multi.c

\begin{algorithmic}[1] 
 \Procedure{t\_aula\_iniciar\_vacia}{$aula$}
   \For{i=0 \textbf{to} ancho del aula}
      \For{j=0 \textbf{to} alto del aula}
        \State{aula.posiciones[i][j] = 0}
      \EndFor
   \EndFor
   \State{aula.cantidad\_de\_personas = 0}
   \State{aula.rescatistas\_disponibles = Cantidad inicial}
 \EndProcedure
\end{algorithmic}
\smallskip
Inicializa un aula vacía con cierta cantidad de rescatistas.
\bigskip

\begin{algorithmic}[1]  
 \Procedure{t\_aula\_ingresar}{$aula,\ alumno$}
   \State{LOCK(mutex\_cantidad\_de\_personas)}
     \IndState{aula.cantidad\_de\_personas ++}
   \State{UNLOCK(mutex\_cantidad\_de\_personas)}
   
   \State{LOCK(mutex\_posiciones[alumno.fila][alumno.columna])}
     \IndState{aula.posiciones[alumno.fila][alumno.columna] ++}
   \State{UNLOCK(mutex\_posiciones[alumno.fila][alumno.columna])}
 \EndProcedure
\end{algorithmic}
\smallskip
Ingresa un alumno al aula en la posición indicada por él. Para que no haya problemas de concurrencia se piden los mutex
que actúan sobre la cantidad de personas del aula y de la posición indicada.
\bigskip

\begin{algorithmic}[1] 
 \Procedure{t\_aula\_liberar}{$aula,\ alumno$}
   \State{LOCK(mutex\_cantidad\_de\_personas)}
     \IndState{aula.cantidad\_de\_personas - -}
   \State{UNLOCK(mutex\_cantidad\_de\_personas)}
 \EndProcedure
\end{algorithmic}
\smallskip
Retira un alumno del aula. Por la misma razón que antes se pide el mutex de la cantidad de personas. %TODO: está bien esto? no lo debería borrar tmb de la posicion en la matriz??
\bigskip

\begin{algorithmic}[1] 
 \Procedure{terminar\_servidor\_de\_alumno}{socket\_fd, aula, alumno}
   \State{close(socket\_fd}
   \State{t\_aula\_liberar(aula, alumno)}
   \State{exit(-1)}
 \EndProcedure
\end{algorithmic}
\smallskip
Termina la conexión con el cliente.
\bigskip

\begin{algorithmic}[1] 
 \Function{intentar\_moverse}{aula, alumno, direccion}
   \State{Calculo la nueva\_posicion = (fila, columna)}
   
   \If{nueva\_posicion es la salida}
     \State{alumno.salio = true}
     \State{pudo\_moverse = true}
   \EndIf
   
   \If{nueva\_posicion esta entre límites \textbf{\&\&} aula.posiciones[nueva\_posicion] < Máx.cant. personas}
     \State{pudo\_moverse = true}
   \EndIf
   \State{vieja\_posicion = (alumno.fila, alumno.columna)}
   
   \If{nueva\_posicion.fila < vieja\_posicion.fila \textbf{||} (nueva\_posicion.fila == vieja\_posicion.fila \textbf{\&\&} nueva\_posicion.columna < vieja\_posicion.columna)}
     \If{!alumno.salio}
       \State{LOCK(mutex\_posiciones[nueva\_posicion.fila][nueva\_posicion.columna])}
     \EndIf
     \State{LOCK(mutex\_posiciones[vieja\_posicion.fila][vieja\_posicion.columna])}
     \State{orden\_locks = 0}
   \Else
     \State{LOCK(mutex\_posiciones[vieja\_posicion.fila][vieja\_posicion.columna])}
     \If{!alumno.salio}
       \State{LOCK(mutex\_posiciones[nueva\_posicion.fila][nueva\_posicion.columna])}
     \EndIf
     \State{orden\_locks = 1}
   \EndIf
   
   \If{pudo\_moverse}
     \If{!alumno.salio}
       \State{aula.posiciones[nueva\_posicion] ++}
     \EndIf
     \State{aula.posiciones[vieja\_posicion] - -}
     \State{actualizar posicion del alumno}
   \EndIf
   
   \If{orden\_locks == 0}
     \State{UNLOCK(mutex\_posiciones[vieja\_posicion.fila][vieja\_posicion.columna])}
     \If{!alumno.salio}
       \State{UNLOCK(mutex\_posiciones[nueva\_posicion.fila][nueva\_posicion.columna])}
     \EndIf
   \Else
     \If{!alumno.salio}
       \State{UNLOCK(mutex\_posiciones[nueva\_posicion.fila][nueva\_posicion.columna])}
     \EndIf
     \State{UNLOCK(mutex\_posiciones[vieja\_posicion.fila][vieja\_posicion.columna])}
   \EndIf
   \State{\textbf{return} pudo\_moverse}
 \EndFunction
\end{algorithmic}
\smallskip
Intenta mover al alumno dentro del aula hacia la dirección indicada. Para esto debe tomar los mutex de la 
posición vieja y nueva (si es que no salió), se toman en determinado orden para evitar espera circular (ver sección deadlock). %TODO: como pongo un 'link' a esa sección?
Luego devuelve si el alumno pudo moverse o no.
\bigskip

\begin{algorithmic}[1] 
 \Procedure{colocar\_mascara}{$aula,\ alumno$}
   \State{alumno.tiene\_mascara = true}
 \EndProcedure
\end{algorithmic} 
\smallskip
Coloca la máscara en el alumno.
\bigskip

\begin{algorithmic}[1] 
 \Procedure{atendedor\_de\_alumno}{$socket\_fd,\ aula,\ salidas,\ alumno$}
   \If{no se pudo recibir el nombre y la posicion}
     \State{terminar\_servidor\_de\_alumno(socket\_fd, NULL, NULL)}
   \EndIf
   \State{t\_aula\_ingresar(aula, alumno)}
   
   \For{ever}
     \If{no se pudo recibir la direccion}
       \State{terminar\_servidor\_de\_alumno(socket\_fd, aula, alumno)}
     \EndIf
     \State{pudo\_moverse = intentar\_moverse(aula, alumno, direccion)}
     \State{enviar\_respuesta(socket\_fd, pudo\_moverse)}     
     \If{alumno.salio}
       \State{break}
     \EndIf
   \EndFor
   
   \Statex
   \State{LOCK(mutex\_rescatistas)}
      \While{aula.rescatistas\_disponibles == 0}
	 \State{COND\_WAIT(cond\_hay\_rescatistas, mutex\_rescatistas)}
      \EndWhile
      
      \IndState{aula.rescatistas\_disponibles - -}
   \State{UNLOCK(mutex\_rescatistas)}
   
   \Statex
   \State{colocar\_mascara(aula, alumno)}
   
   \Statex
   \State{LOCK(mutex\_rescatistas)}
     \IndState{aula.rescatistas\_disponibles ++}
     \IndState{COND\_SIGNAL(cond\_hay\_rescatistas)}
   \State{UNLOCK(mutex\_rescatistas)}
   
   \Statex
   \State{LOCK(mutex\_pasillo)}
     \IndState{t\_aula\_liberar(aula, alumno)}
     \IndState{salidas.cant\_personas\_pasillo ++}
   \State{UNLOCK(mutex\_pasillo)}
   
   \Statex
   \State{LOCK(mutex\_grupo)}
     \While{salidas.cant\_personas\_grupo >= 5}
        \State{COND\_WAIT(cond\_grupo\_lleno, mutex\_grupo)}
     \EndWhile
     
     \State{LOCK(mutex\_pasillo)}
       \IndState{salidas.cant\_personas\_pasillo - -}
       \IndState{salidas.cant\_personas\_grupo ++}
     \State{UNLOCK(mutex\_pasillo)}
     
     \If{salidas.cant\_personas\_grupo == 5 \textbf{||} (salidas.cant\_personas\_pasillo == 0 \textbf{\&\&} aula.cantidad\_de\_personas == 0)}
       \State{COND\_BROADCAST(cond\_estan\_evacuando)}
     \Else
       \State{COND\_WAIT(cond\_estan\_evacuando, mutex\_grupo)}
     \EndIf
     
     \State{salidas.cant\_personas\_grupo - -}
     \If{salidas.cant\_personas\_grupo > 0}
       \State{\State{COND\_WAIT(cond\_salimos\_todos, mutex\_grupo)}}
     \Else
       \State{COND\_BROADCAST(cond\_salimos\_todos)}
       \State{COND\_BROADCAST(cond\_grupo\_lleno)}
     \EndIf
   \State{UNLOCK(mutex\_grupo)}
   
   \Statex
   \State{enviar\_respuesta(socket\_fd, LIBRE)}
 \EndProcedure
\end{algorithmic} 
\smallskip
Si no se puede recibir el nombre y la posicion se supone una conexión fallida y se corta dicha conexión.
Luego se reciben direcciones hasta que el alumno logra salir del aula. En caso de no recibir una dirección, 
de nuevo, se supone una conexión fallida y se corta dicha conexión. \\

Cuando llega a la salida queda a la espera de un rescatista. En cuanto consigue uno le es colocada la máscara y luego
el rescatista vuelve a quedar libre. Notar que para sincronizar la variable rescatistas\_disponibles 
es pedido el mutex antes y después de modificarla, además en vez de esperar de forma activa ($busy\ waiting$)
utilizamos una variable de condición que despierta cuándo hay rescatistas disponibles.\\

Una vez que el alumno tiene la máscara puesta, entra a un lugar entre el aula y la salida al cuál decidimos llamar 
\textquoteleft pasillo\textquoteright. Recién entonces decimos que el alumno salió efectivamente del aula (linea 28).
La cantidad de personas en el pasillo también está sincronizada con un mutex.\\

En este momento el alumno debe esperar a que haya lugar en el grupo de evacuación. De nuevo utilizamos una variable
de condición que despierta si esto pasa. Ahora el alumno pasa del pasillo al grupo, si son 5 o es el último en salir
avisa a los demás que pueden evacuar. De lo contrario queda a la espera de la señal.\\
De a uno se retiran del grupo (se descuentan de la cantidad de personas en el grupo, variable que está sincronizada).
En cuanto todos se hayan retirado del grupo salen del edificio juntos.
\bigskip

\begin{algorithmic}[1] 
 \Function{main}{void}
   \State{Inicializar conexion con sockets y permitir 5 conexiones en espera}
   \State{t\_aula\_iniciar\_vacia(aula)}
   \State{Inicializar mutexes}
   \State{Inicializar variables de condicion}
   \For{ever}\Comment{Loop de atencion al cliente}
     \If{Se conecto un nuevo cliente}
       \State{Inicializar pthread y sus parametros}
       \State{Crear pthread con la funcion 'atendedor\_de\_alumno'}
     \EndIf
   \EndFor
 \EndFunction
\end{algorithmic}
\smallskip
Crea un nuevo pthread por cada cliente que se conecta al servidor, es decir, uno por cada alumno que ingrese al aula.
Lo crea con la función atendedor\_de\_alumno la cuál está explicada más arriba.