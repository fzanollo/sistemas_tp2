%algorithm de servidor_multi y luego la explicación de cada función, explicación del pasillo y como se retiran los grupos.
Código en servidor\_multi.c
\begin{algorithmic}[1] 
 \Procedure{t\_aula\_iniciar\_vacia}{$aula$}
   \For{i=0 \textbf{to} ancho del aula}
      \For{j=0 \textbf{to} alto del aula}
	 \State{aula.posiciones[i][j] = 0}
      \EndFor
   \EndFor
   \State{aula.cantidad\_de\_personas = 0}
   \State{aula.rescatistas\_disponibles = Cantidad inicial}
 \EndProcedure
 \bigskip
 
 
 \Procedure{t\_aula\_ingresar}{$aula,\ alumno$}
   \State{LOCK(mutex\_cantidad\_de\_personas)}
     \IndState{aula.cantidad\_de\_personas ++}
   \State{UNLOCK(mutex\_cantidad\_de\_personas)}
   
   \State{LOCK(mutex\_posiciones[alumno.fila][alumno.columna])}
     \IndState{aula.posiciones[alumno.fila][alumno.columna] ++}
   \State{UNLOCK(mutex\_posiciones[alumno.fila][alumno.columna])}
 \EndProcedure
 \bigskip
 
 
 \Procedure{t\_aula\_liberar}{$aula,\ alumno$}
   \State{LOCK(mutex\_cantidad\_de\_personas)}
     \IndState{aula.cantidad\_de\_personas - -}
   \State{UNLOCK(mutex\_cantidad\_de\_personas)}
 \EndProcedure
 \bigskip
 
 
 \Procedure{terminar\_servidor\_de\_alumno}{socket\_fd, aula, alumno}
   \State{close(socket\_fd}
   \State{t\_aula\_liberar(aula, alumno)}
   \State{exit(-1)}
 \EndProcedure
 \bigskip
 
 
 \Function{intentar\_moverse}{aula, alumno, direccion}
   \State{Calculo la nueva\_posicion = (fila, columna)}
   
   \If{nueva\_posicion es la salida}
     \State{alumno.salio = true}
     \State{pudo\_moverse = true}
   \EndIf
   
   \If{nueva\_posicion esta entre límites \textbf{\&\&} aula.posiciones[nueva\_posicion] < Máx.cant. personas}
     \State{pudo\_moverse = true}
   \EndIf
   \State{vieja\_posicion = (alumno.fila, alumno.columna)}
   
   \If{nueva\_posicion.fila < vieja\_posicion.fila \textbf{||} (nueva\_posicion.fila == vieja\_posicion.fila \textbf{\&\&} nueva\_posicion.columna < vieja\_posicion.columna)}
     \If{!alumno.salio}
       \State{LOCK(mutex\_posiciones[nueva\_posicion.fila][nueva\_posicion.columna])}
     \EndIf
     \State{LOCK(mutex\_posiciones[vieja\_posicion.fila][vieja\_posicion.columna])}
     \State{orden\_locks = 0}
   \Else
     \State{LOCK(mutex\_posiciones[vieja\_posicion.fila][vieja\_posicion.columna])}
     \If{!alumno.salio}
       \State{LOCK(mutex\_posiciones[nueva\_posicion.fila][nueva\_posicion.columna])}
     \EndIf
     \State{orden\_locks = 1}
   \EndIf
   
   \If{pudo\_moverse}
     \If{!alumno.salio}
       \State{aula.posiciones[nueva\_posicion] ++}
     \EndIf
     \State{aula.posiciones[vieja\_posicion] - -}
     \State{actualizar posicion del alumno}
   \EndIf
   
   \If{orden\_locks == 0}
     \State{UNLOCK(mutex\_posiciones[vieja\_posicion.fila][vieja\_posicion.columna])}
     \If{!alumno.salio}
       \State{UNLOCK(mutex\_posiciones[nueva\_posicion.fila][nueva\_posicion.columna])}
     \EndIf
   \Else
     \If{!alumno.salio}
       \State{UNLOCK(mutex\_posiciones[nueva\_posicion.fila][nueva\_posicion.columna])}
     \EndIf
     \State{UNLOCK(mutex\_posiciones[vieja\_posicion.fila][vieja\_posicion.columna])}
   \EndIf
   \State{\textbf{return} pudo\_moverse}
 \EndFunction
 \bigskip
 
 
 \Procedure{colocar\_mascara}{$aula,\ alumno$}
   \State{alumno.tiene\_mascara = true}
 \EndProcedure
 \bigskip
 
 
 \Procedure{atendedor\_de\_alumno}{$socket\_fd,\ aula,\ salidas,\ alumno$}
   \If{no se pudo recibir el nombre y la posicion}
     \State{terminar\_servidor\_de\_alumno(socket\_fd, NULL, NULL)}
   \EndIf
   \State{t\_aula\_ingresar(aula, alumno)}
   
   \For{Ever}
     \If{no se pudo recibir la direccion}
       \State{terminar\_servidor\_de\_alumno(socket\_fd, aula, alumno)}
     \EndIf
     \State{pudo\_moverse = intentar\_moverse(aula, alumno, direccion)}
     \State{enviar\_respuesta(socket\_fd, pudo\_moverse)}     
     \If{alumno.salio}
       \State{break}
     \EndIf
   \EndFor
   
   \State{LOCK(mutex\_rescatistas)}
      \While{aula.rescatistas\_disponibles == 0}
	 \State{COND\_WAIT(cond\_hay\_rescatistas, mutex\_rescatistas)}
      \EndWhile
      
      \IndState{aula.rescatistas\_disponibles - -}
   \State{UNLOCK(mutex\_rescatistas)}
   
   \State{colocar\_mascara(aula, alumno)}
   
   \State{LOCK(mutex\_rescatistas)}
     \State{aula.rescatistas\_disponibles ++}
     \State{COND\_SIGNAL(cond\_hay\_rescatistas)}
   \State{UNLOCK(mutex\_rescatistas)}
   
   \State{LOCK(mutex\_pasillo)}
     \State{t\_aula\_liberar(aula, alumno)}
     \State{salidas.cant\_personas\_pasillo ++}
   \State{UNLOCK(mutex\_pasillo)}
   
   \State{LOCK(mutex\_grupo)}
     \While{salidas.cant\_personas\_grupo >= 5}
	 \State{COND\_WAIT(cond\_grupo\_lleno, mutex\_grupo)}
     \EndWhile
     
     \State{LOCK(mutex\_pasillo)}
       \State{salidas.cant\_personas\_pasillo - -}
       \State{salidas.cant\_personas\_grupo ++}
     \State{UNLOCK(mutex\_pasillo)}
     
     \If{salidas.cant\_personas\_grupo == 5 \textbf{||} (salidas.cant\_personas\_pasillo == 0 \textbf{\&\&} aula.cantidad\_de\_personas == 0)}
       \State{COND\_BROADCAST(cond\_estan\_evacuando)}
     \Else
       \State{COND\_WAIT(cond\_estan\_evacuando, mutex\_grupo)}
     \EndIf
     
     \State{salidas.cant\_personas\_grupo - -}
     \If{salidas.cant\_personas\_grupo > 0}
       \State{\State{COND\_WAIT(cond\_salimos\_todos, mutex\_grupo)}}
     \Else
       \State{COND\_BROADCAST(cond\_salimos\_todos)}
       \State{COND\_BROADCAST(cond\_grupo\_lleno)}
     \EndIf
   \State{UNLOCK(mutex\_grupo)}
   
   \State{enviar\_respuesta(socket\_fd, LIBRE)}
 \EndProcedure
 \bigskip
 
 
 \Function{main}{void}
   \State{Inicializar conexion con sockets y permitir 5 conexiones en espera}
   \State{t\_aula\_iniciar\_vacia(aula)}
   \State{Inicializar mutexes}
   \State{Inicializar variables de condicion}
   \For{Ever}\Comment{Loop de atencion al cliente}
     \If{Se conecto un nuevo cliente}
       \State{Inicializar pthread y sus parametros}
       \State{Crear pthread con la funcion 'atendedor\_de\_alumno'}
     \EndIf
   \EndFor
 \EndFunction
 
\end{algorithmic}

