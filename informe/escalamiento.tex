En caso de un aumento drástico en la cantidad de alumnos el problema a tener en cuenta es la memoria.\\

Un pthread es creado por cada alumno (o cliente) que se conecta al servidor. Esto implica un nuevo stack, 
el tamaño del stack es variable. El valor por defecto sobre un sistema de 32 bits (y lo vamos a asumir para este caso) es de 2MB.\cite{manUnix}\\ 
%TODO add this to bibliography
%from man pthread_create ``On Linux/x86-32,  the  default  stack  size  for  a  new  thread  is  2 megabytes.''

Entonces para un servidor con 10 clientes (10 pthreads) se necesita\footnote{En realidad si la memoria no alcanza se podría usar $swapping$ pero esto generaría otro problema: $thrashing$ (se cargan y descargan sucesiva y constantemente partes de la imagen de un proceso desde y hacia la memoria virtual)}
una memoria de $10\ *\ 2$ MB = 20 MB.\\
Y para un servidor con 1000 clientes se necesita una memoria de 2GB.\\

La solución obvia mediante hardware es agregar más memoria principal al servidor de manera que alcance para la cantidad de clientes.
Esto conlleva un costo muy alto y puede llegar a volverse inmanejable.\\

Soluciones mediante software:
\begin{description}
 \item \textbf{Sin pthreads:} Se podría implementar un servidor capaz de atender a distintos clientes en \textquoteleft simultaneo\textquoteright.\\
 Con un mecanismo de scheduling tal y como se simula paralelismo de procesos contando con solo un CPU.  
 El problema de esta solución es que puede tardar mucho en atender a los clientes (depende estrictamente de la cantidad y el mecanismo usado).
 Incluso podría haber $starvation$.
 
 \item \textbf{Con pthreads:} Existe una solución simple que no modifica la implementación original y otra, compleja, que si lo hace.
 
    \begin{description} 
    \item \textbf{Simple:} El tamaño del stack de cada pthread se puede modificar. Entonces se podría utilizar un stack más pequeño para no tener
    tantos problemas de memoria. El inconveniente de esta técnica es que se debe definir un tamaño de stack apropiado, debe tener espacio para contener
    todas las variables privadas del pthread.
    
    \item \textbf{Compleja:} En vez de tener un pthread por cada cliente, se pueden compartir. Si se crea un pthread que atiende a X cantidad de clientes 
    entonces la cantidad total de memoria requerida para N clientes es de $\frac{N}{X}\ *\ 2\ MB$.\\
    Se debe elegir un X adecuado ya que si es muy chico se consume más memoria pero los clientes se atienden más rápido y viceversa de ser muy grande.\\
    Por ejemplo para X = 10 la cantidad total de memoria requerida para 1000 clientes es de $\frac{1000}{10}\ *\ 2$ MB = 200 MB. Se reduce en un 90\%.
    \end{description}
    
\end{description}