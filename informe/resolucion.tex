%Justificar en todos los casos todas las decisiones tomadas.
%Implementación realizada: indicando los archivos creados y/o modificados. Dificultades encontradas
%Qué tipos de datos usamos para sincronizar los procesos. Por qué
\subsection{Variables utilizadas}
\subsubsection{Compartidas}
\textbf{t\_aula aula.}
Estructura que contiene: 
\begin{itemize}
 \item Una matríz de enteros donde cada posición equivale a $1\ m^{2}$ del aula.
 \item Cantidad de personas dentro del aula.
 \item Cantidad de rescatístas.
\end{itemize}

\subsubsection{Privadas}
\textbf{t\_persona alumno.}
Estructura que contiene: 
\begin{itemize}
 \item Nombre
 \item Posición en la que se encuentra (fila y columna)
 \item Un booleano indicando si salió y otro si tiene la máscara.
\end{itemize}

\smallskip
\textbf{t\_salidas salidas.}
Estructura que contiene: 
\begin{itemize}
 \item Cantidad de personas en el pasillo. 
 El pasillo es una zona intermedia entre el aula y el grupo. En él se encuentran 
 los alumnos con máscara esperando para poder ingresar en un grupo.
 \item Cantidad de personas en el grupo. 
 En el grupo los alumnos esperan ser cinco para poder ser evacuados.
\end{itemize}

\subsubsection{Caso especial}
\textbf{t\_parametros.} 
Estructura que contiene: 
\begin{itemize}
 \item Identificador del socket que utilizará el pthread para comunicarse con el cliente.
 Este socket es único para cada pthread.
 \item Variable compartida $el\_ aula$.
 \item Variable privada $salidas$.
\end{itemize}

La implementación de pthreads sólo admite el pasaje de un parámetro y debe
ser una estructura, por eso t\_parametros es necesaria. Al crearla utilizamos malloc para alojarla en memoria, 
de no hacerlo de esta forma la variable \textquoteleft muere\textquoteright al final del $scope$ y
el pthread estaría accediendo a posiciones inválidas (contienen basura).

\subsection{Mutexes y Variables de Condición}
Para sincronizar los procesos utilizamos:

\begin{itemize}
\item Mutexes
  \medskip

  \begin{tabular}{|p{7.5cm}|p{9cm}|}
  \hline
  Nombre del mutex & Para control de acceso a... \\
  \hline
  mutex\_posiciones (matriz de tamaño ancho del aula por alto del aula) & las posiciones del aula. \\
  \hline
  mutex\_cantidad\_de\_personas & la cantidad de personas en el aula. \\
  \hline
  mutex\_rescatistas & la cantidad de rescatistas en el aula. \\
  \hline
  mutex\_pasillo & la cantidad de personas en el pasillo, es decir, fuera del aula (con máscara) pero no en el grupo. \\
  \hline
  mutex\_grupo & la cantidad de personas en el grupo, es decir, esperando formar grupo de 5 para salir. \\
  \hline
  \end{tabular}

\item Variables de condición
  \medskip

  \begin{tabular}{|l|p{5cm}|p{7cm}|}
  \hline
  Nombre de la variable & Condición de bloqueo & Explicación  \\
  \hline
  cond\_hay\_rescatistas & Cantidad de rescatistas disponibles = 0 & Espera a que haya un rescatista libre. \\ %el_aula->rescatistas_disponibles == 0
  \hline
  cond\_grupo\_lleno & Cantidad de personas en el grupo >= 5 & Espera a que haya espacio en el grupo de evacuacion. \\ %salidas->cant_personas_grupo >= 5
  \hline
  cond\_estan\_evacuando & Cantidad de personas en el grupo $\neq$ 5 y hay más gente en el aula o en el pasillo & Espera a que sean 5 en el grupo de 
    evacuación a menos que sea la última persona en salir. \\ %TODO: ver si esta bien la condicion de bloqueo %salidas->cant_personas_grupo == 5 || (salidas->cant_personas_pasillo == 0 && el_aula->cantidad_de_personas == 0)
  \hline
  cond\_salimos\_todos & Cantidad de personas en el grupo > 0 & Espera a que termine de salir todo el grupo. \\ %salidas->cant_personas_grupo > 0
  \hline
  \end{tabular}

\end{itemize}

\subsection{Implementación}
Código del archivo servidor\_multi.c

\begin{algorithmic}[1] 
 \Procedure{t\_aula\_iniciar\_vacia}{aula}
   \For{i=0 \textbf{to} ancho del aula}
      \For{j=0 \textbf{to} alto del aula}
        \State{aula.posicion[i][j] = 0}
      \EndFor
   \EndFor
   \State{aula.cantidad\_de\_personas = 0}
   \State{aula.rescatistas\_disponibles = Cantidad inicial}
 \EndProcedure
\end{algorithmic}
\smallskip
Inicializa un aula vacía con cierta cantidad de rescatistas.
\bigskip

\begin{algorithmic}[1]  
 \Procedure{t\_aula\_ingresar}{aula,\ alumno}
   \State{LOCK(mutex\_cantidad\_de\_personas)}
     \IndState{aula.cantidad\_de\_personas ++}
   \State{UNLOCK(mutex\_cantidad\_de\_personas)}
   
   \State{LOCK(mutex\_posiciones[alumno.fila][alumno.columna])}
     \IndState{aula.posicion[alumno.fila][alumno.columna] ++}
   \State{UNLOCK(mutex\_posiciones[alumno.fila][alumno.columna])}
 \EndProcedure
\end{algorithmic}
\smallskip
Ingresa un alumno al aula en la posición indicada por él. Para que no haya problemas de concurrencia se piden los mutex
que actúan sobre la cantidad de personas del aula y de la posición indicada.
\bigskip

\begin{algorithmic}[1] 
 \Procedure{t\_aula\_liberar}{aula,\ alumno}
   \State{LOCK(mutex\_cantidad\_de\_personas)}
     \IndState{aula.cantidad\_de\_personas - -}
   \State{UNLOCK(mutex\_cantidad\_de\_personas)}
 \EndProcedure
\end{algorithmic}
\smallskip
Retira un alumno del aula. Por la misma razón que antes se pide el mutex de la cantidad de personas.
\bigskip

\begin{algorithmic}[1] 
 \Procedure{terminar\_servidor\_de\_alumno}{socket\_fd, aula, alumno}
   \State{close(socket\_fd)}
   \State{t\_aula\_liberar(aula, alumno)}
   \State{EXIT()} \Comment{Cierra el pthread.}
 \EndProcedure
\end{algorithmic}
\smallskip
Termina la conexión con el cliente.
\bigskip

\begin{algorithmic}[1] 
 \Function{intentar\_moverse}{aula, alumno, direccion}
   \State{Calculo la nueva\_posicion = (fila, columna)}
   
   \If{nueva\_posicion es la salida}
     \State{alumno.salio = true}
     \State{pudo\_moverse = true}
   \EndIf
   
   \State{LOCK(mutex\_posiciones[nueva\_posicion])}
   \If{nueva\_posicion esta entre límites \textbf{and} aula.posicion[nueva\_posicion] < Máx.cant. personas}
     \State{pudo\_moverse = true}
   \EndIf
   \State{UNLOCK(mutex\_posiciones[nueva\_posicion])}
   \Statex
   \State{vieja\_posicion = (alumno.fila, alumno.columna)}
   \Statex
   \If{pudo\_moverse}
      \If{nueva\_posicion.fila < vieja\_posicion.fila \textbf{or} (nueva\_posicion.fila == vieja\_posicion.fila\\ 
      \hspace{30pt}\textbf{and} nueva\_posicion.columna < vieja\_posicion.columna)}
	\If{!alumno.salio}
	  \State{LOCK(mutex\_posiciones[nueva\_posicion.fila][nueva\_posicion.columna])}
	\EndIf
	\State{LOCK(mutex\_posiciones[vieja\_posicion.fila][vieja\_posicion.columna])}
	\State{orden\_locks = 0}
      \Else
	\State{LOCK(mutex\_posiciones[vieja\_posicion.fila][vieja\_posicion.columna])}
	\If{!alumno.salio}
	  \State{LOCK(mutex\_posiciones[nueva\_posicion.fila][nueva\_posicion.columna])}
	\EndIf
	\State{orden\_locks = 1}
      \EndIf
   
     \If{!alumno.salio}
       \State{aula.posicion[nueva\_posicion] ++}
     \EndIf
     \State{aula.posicion[vieja\_posicion] - -}
     \State{actualizar posicion del alumno}
      
      \If{orden\_locks == 0}
	\State{UNLOCK(mutex\_posiciones[vieja\_posicion.fila][vieja\_posicion.columna])}
	\If{!alumno.salio}
	  \State{UNLOCK(mutex\_posiciones[nueva\_posicion.fila][nueva\_posicion.columna])}
	\EndIf
      \Else
	\If{!alumno.salio}
	  \State{UNLOCK(mutex\_posiciones[nueva\_posicion.fila][nueva\_posicion.columna])}
	\EndIf
	\State{UNLOCK(mutex\_posiciones[vieja\_posicion.fila][vieja\_posicion.columna])}
      \EndIf
   \EndIf
   \State{\textbf{return} pudo\_moverse}
 \EndFunction
\end{algorithmic}
\smallskip
Intenta mover al alumno dentro del aula hacia la dirección indicada. Para esto debe tomar los mutex de la 
posición vieja y nueva (si es que no salió), se toman en determinado orden para evitar espera circular (ver sección \ref{sec:deadlock}).
Luego devuelve si el alumno pudo moverse o no.
\bigskip

\begin{algorithmic}[1] 
 \Procedure{colocar\_mascara}{aula,\ alumno}
   \State{alumno.tiene\_mascara = true}
 \EndProcedure
\end{algorithmic} 
\smallskip
Coloca la máscara en el alumno.
\bigskip

\begin{algorithmic}[1] 
 \Procedure{atendedor\_de\_alumno}{socket\_fd,\ aula,\ salidas,\ alumno}
   \If{no se pudo recibir el nombre y la posicion}
     \State{terminar\_servidor\_de\_alumno(socket\_fd, NULL, NULL)}
   \EndIf
   \State{t\_aula\_ingresar(aula, alumno)}
   
   \For{ever}
     \If{no se pudo recibir la direccion}
       \State{terminar\_servidor\_de\_alumno(socket\_fd, aula, alumno)}
     \EndIf
     \State{pudo\_moverse = intentar\_moverse(aula, alumno, direccion)}
     \State{enviar\_respuesta(socket\_fd, pudo\_moverse)}     
     \If{alumno.salio}
       \State{break}
     \EndIf
   \EndFor
   
   \Statex
   \State{LOCK(mutex\_rescatistas)}
      \While{aula.rescatistas\_disponibles == 0}
	 \State{COND\_WAIT(cond\_hay\_rescatistas, mutex\_rescatistas)}
      \EndWhile
      
      \IndState{aula.rescatistas\_disponibles - -}
   \State{UNLOCK(mutex\_rescatistas)}
   
   \Statex
   \State{colocar\_mascara(aula, alumno)}
   
   \Statex
   \State{LOCK(mutex\_rescatistas)}
     \IndState{aula.rescatistas\_disponibles ++}
   \State{UNLOCK(mutex\_rescatistas)}
   \State{COND\_SIGNAL(cond\_hay\_rescatistas)}
   
   \Statex
   \State{LOCK(mutex\_pasillo)}
     \IndState{t\_aula\_liberar(aula, alumno)}
     \IndState{salidas.cant\_personas\_pasillo ++}
   \State{UNLOCK(mutex\_pasillo)}
   
   \Statex
   \State{LOCK(mutex\_grupo)}
     \While{salidas.cant\_personas\_grupo >= 5}
        \State{COND\_WAIT(cond\_grupo\_lleno, mutex\_grupo)}
     \EndWhile
     
     \State{LOCK(mutex\_pasillo)}
       \IndState{salidas.cant\_personas\_pasillo - -}
       \IndState{salidas.cant\_personas\_grupo ++}
     \State{UNLOCK(mutex\_pasillo)}
     
     \State{LOCK(mutex\_pasillo)}
     \State{LOCK(mutex\_cantidad\_de\_personas)}
     \If {salidas.cant\_personas\_grupo == 5 \textbf{or} (salidas.cant\_personas\_pasillo == 0\\
     \hspace{\algorithmicindent}\textbf{and} aula.cantidad\_de\_personas == 0)}
       \State{UNLOCK(mutex\_cantidad\_de\_personas)}
       \State{UNLOCK(mutex\_pasillo)}
       \State{COND\_BROADCAST(cond\_estan\_evacuando)}
     \Else
       \State{UNLOCK(mutex\_cantidad\_de\_personas)}
       \State{UNLOCK(mutex\_pasillo)}
       \State{COND\_WAIT(cond\_estan\_evacuando, mutex\_grupo)}
     \EndIf
     
     \State{salidas.cant\_personas\_grupo - -}
     \If{salidas.cant\_personas\_grupo > 0}
       \State{\State{COND\_WAIT(cond\_salimos\_todos, mutex\_grupo)}}
     \Else
       \State{COND\_BROADCAST(cond\_salimos\_todos)}
       \State{COND\_BROADCAST(cond\_grupo\_lleno)}
     \EndIf
   \State{UNLOCK(mutex\_grupo)}
   
   \Statex
   \State{enviar\_respuesta(socket\_fd, LIBRE)}
 \EndProcedure
\end{algorithmic} 
\smallskip
Si no se puede recibir el nombre y la posicion se supone una conexión fallida y se corta dicha conexión.
Luego se reciben direcciones hasta que el alumno logra salir del aula. En caso de no recibir una dirección, 
de nuevo, se supone una conexión fallida y se corta dicha conexión. \\

Cuando llega a la salida queda a la espera de un rescatista. En cuanto consigue uno le es colocada la máscara y luego
el rescatista vuelve a quedar libre. Notar que para sincronizar la variable rescatistas\_disponibles 
es pedido el mutex antes y después de modificarla, además en vez de esperar de forma activa ($busy\ waiting$)
utilizamos una variable de condición que despierta cuándo hay rescatistas disponibles.\\

Una vez que el alumno tiene la máscara puesta, entra a un lugar entre el aula y la salida al cuál decidimos llamar 
\textquoteleft pasillo\textquoteright. Recién entonces decimos que el alumno salió efectivamente del aula (linea 28).
La cantidad de personas en el pasillo también está sincronizada con un mutex.\\

En este momento el alumno debe esperar a que haya lugar en el grupo de evacuación. De nuevo utilizamos una variable
de condición que despierta si esto pasa. Ahora el alumno pasa del pasillo al grupo, si son 5 o es el último en salir
avisa a los demás que pueden evacuar. De lo contrario queda a la espera de la señal.\\
De a uno se retiran del grupo (se descuentan de la cantidad de personas en el grupo, variable que está sincronizada).
En cuanto todos se hayan retirado del grupo salen del edificio juntos.
\bigskip

\begin{algorithmic}[1] 
 \Function{main}{void}
   \State{Inicializar conexion con sockets y permitir 5 conexiones en espera}
   \State{t\_aula\_iniciar\_vacia(aula)}
   \State{Inicializar mutexes}
   \State{Inicializar variables de condicion}
   \For{ever}\Comment{Loop de atencion al cliente}
     \If{Se conecto un nuevo cliente}
       \State{Inicializar pthread y sus parametros}
       \State{Crear pthread con la funcion 'atendedor\_de\_alumno'}
     \EndIf
   \EndFor
 \EndFunction
\end{algorithmic}
\smallskip
Crea un nuevo pthread por cada cliente que se conecta al servidor, es decir, uno por cada alumno que ingrese al aula.
Lo crea con la función atendedor\_de\_alumno la cuál está explicada más arriba.

\subsection{Deadlock}
\label{sec:deadlock}
Un grupo de procesos están en estado de $deadlock$ si cada uno de ellos está esperando un evento que sólo otro proceso del grupo puede causar.
Vamos a analizar el código para demostrar que está libre de deadlocks.

Antes de eso veremos cuáles son las condiciones que debe cumplir un sistema para tener la posibilidad de llegar a un estado de deadlock, llamadas
condiciones de Coffman:
\begin{itemize}
 \item Exclusión mutua: cada recurso está asignado exactamente a un proceso o está disponible.
 \item Hold-and-wait: Los procesos que tienen asignado un recurso pueden requerir otro/s recurso/s.
 \item No-preemption: Los recursos asignados a procesos no pueden ser removidos por la fuerza.
 \item Espera circular: Debe haber una lista de dos o más procesos, cada uno de ellos esperando un recurso que tiene el anterior.
\end{itemize}

En nuestra implementación no se cumple la última condición de Coffman, por ende está libre de deadlocks.
Entonces veamos por qué no hay espera circular:

Para asegurarnos de que no haya espera circular en cuanto a las posiciones del aula lo que hicimos fue disponer un orden de bloqueos.
%TODO: agregar gráfico o explicación del orden
De esta forma, si dos clientes deben bloquear las posiciones 1 y 2. Ambos lo van a hacer en el mismo orden. Entonces si 1 < 2, uno de ellos tomará el lock
de la posición 1 y el otro se quedará esperando a que se libere. Núnca sucede que uno de ellos toma el 2 sin tener el 1.

Otro momento donde hay hold and wait es cuando el cliente tiene el mutex del grupo y pide el del pasillo. Pero no hay ningún lugar en el código donde 
pueda llegar a pedir primero el pasillo y luego el grupo. Notar que sí hay un pedido del mutex del pasillo sin el del grupo, pero en ese caso
el cliente eventualmente desbloqueará el pasillo, permitiendo a los demás seguir su curso.
%....................TODO:
Grupo-pasillo-aula, explicar porq no hay deadlock con eso
Más grafiquitos, los está haciendo Dami
%....................

Una variante de deadlock es livelock, ambos son formas de inanición. Excepto que el estado de los procesos envueltos en el livelock está cambiando constantemente,
aunque sin avanzar. Al suponer que los clientes son personas lógicas evitamos caer en un livelock, ya que no tiene sentido moverse en círculos dentro del aula
en vez de salir. %TODO: esto está horriblemente horrible, aprendé a redactar!

\subsection{Paralelismo}
El objetivo de esta implementación es maximizar el grado de paralelismo, por esta razón decidimos utilizar semáforos independientes para las posiciones
del aula y sus atributos.

En un principio nuestra solución utilizaba un unico mutex para restringir el acceso a la estructura del aula, si bien esta solución funcionaba
no se permitian movimientos paralelos dentro del aula, aunque esas posiciones fueran totalmente distintas, lo cual quita paralelismo.

Además se buscó reducir las secciones críticas lo mayor posible. Así, por ejemplo, mientras a un alumno le están poniendo una máscara otro puede bloquear el mutex
de los rescatistas ya que no está siendo modificada la cantidad de ellos.

