%Justificar en todos los casos todas las decisiones tomadas.
%Implementación realizada: indicando los archivos creados y/o modificados. Dificultades encontradas
%Qué tipos de datos usamos para sincronizar los procesos. Por qué
\subsection{Variables utilizadas}
\subsubsection{Compartidas}
\textbf{t\_aula aula.}
Estructura que contiene: 
\begin{itemize}
 \item Una matríz de enteros donde cada posición equivale a $1\ m^{2}$ del aula.
 \item Cantidad de personas dentro del aula.
 \item Cantidad de rescatístas.
\end{itemize}

\subsubsection{Privadas}
\textbf{t\_persona alumno.}
Estructura que contiene: 
\begin{itemize}
 \item Nombre
 \item Posición en la que se encuentra (fila y columna)
 \item Un booleano indicando si salió y otro si tiene la máscara.
\end{itemize}

\smallskip
\textbf{t\_salidas salidas.}
Estructura que contiene: 
\begin{itemize}
 \item Cantidad de personas en el pasillo. 
 El pasillo es una zona intermedia entre el aula y el grupo. En él se encuentran 
 los alumnos con máscara esperando para poder ingresar en un grupo.
 \item Cantidad de personas en el grupo. 
 En el grupo los alumnos esperan ser cinco para poder ser evacuados.
\end{itemize}

\subsubsection{Caso especial}
\textbf{t\_parametros.} 
Estructura que contiene: 
\begin{itemize}
 \item Identificador del socket que utilizará el pthread para comunicarse con el cliente.
 Este socket es único para cada pthread.
 \item Variable compartida $el\_ aula$.
 \item Variable privada $salidas$.
\end{itemize}

La implementación de pthreads sólo admite el pasaje de un parámetro y debe
ser una estructura, por eso t\_parametros es necesaria. Al crearla utilizamos malloc para alojarla en memoria, 
de no hacerlo de esta forma la variable \textquoteleft muere\textquoteright al final del $scope$ y
el pthread estaría accediendo a posiciones inválidas (contienen basura).

\subsection{Mutexes y Variables de Condición}
Para sincronizar los procesos utilizamos:

\begin{itemize}
\item Mutexes
  \medskip

  \begin{tabular}{|p{7.5cm}|p{9cm}|}
  \hline
  \textbf{Nombre del mutex} & \textbf{Para control de acceso a...} \\
  \hline
  mutex\_posiciones (matriz de tamaño ancho del aula por alto del aula) & las posiciones del aula. \\
  \hline
  mutex\_cantidad\_de\_personas & la cantidad de personas en el aula. \\
  \hline
  mutex\_rescatistas & la cantidad de rescatistas en el aula. \\
  \hline
  mutex\_pasillo & la cantidad de personas en el pasillo, es decir, fuera del aula (con máscara) pero no en el grupo. \\
  \hline
  mutex\_grupo & la cantidad de personas en el grupo, es decir, esperando formar grupo de 5 para salir. \\
  \hline
  \end{tabular}

\item Variables de condición
  \medskip

  \begin{tabular}{|l|p{5cm}|p{7cm}|}
  \hline
  \textbf{Nombre de la variable} & \textbf{Condición de bloqueo} & \textbf{Explicación}  \\
  \hline
  cond\_hay\_rescatistas & Cantidad de rescatistas disponibles = 0 & Espera a que haya un rescatista libre. \\ %el_aula->rescatistas_disponibles == 0
  \hline
  cond\_grupo\_lleno & Cantidad de personas en el grupo >= 5 & Espera a que haya espacio en el grupo de evacuación. \\ %salidas->cant_personas_grupo >= 5
  \hline
  cond\_estan\_evacuando & Cantidad de personas en el grupo $\neq$ 5 y hay más gente en el aula o en el pasillo & Espera a que sean 5 en el grupo de 
    evacuación a menos que sea la última persona en salir. \\ %TODO: ver si esta bien la condicion de bloqueo %salidas->cant_personas_grupo == 5 || (salidas->cant_personas_pasillo == 0 && el_aula->cantidad_de_personas == 0)
  \hline
  cond\_salimos\_todos & Cantidad de personas en el grupo > 0 & Espera a que termine de salir todo el grupo. \\ %salidas->cant_personas_grupo > 0
  \hline
  \end{tabular}

\end{itemize}

\subsection{Implementación}
Código en servidor\_multi.c
\begin{algorithmic}[1] 
 \Procedure{t\_aula\_iniciar\_vacia}{$aula$}
   \For{i=0 \textbf{to} ancho del aula}
      \For{j=0 \textbf{to} alto del aula}
	 \State{aula.posiciones[i][j] = 0}
      \EndFor
   \EndFor
   \State{aula.cantidad\_de\_personas = 0}
   \State{aula.rescatistas\_disponibles = Cantidad inicial}
 \EndProcedure
 \bigskip
 
 
 \Procedure{t\_aula\_ingresar}{$aula,\ alumno$}
   \State{LOCK(mutex\_cantidad\_de\_personas)}
     \IndState{aula.cantidad\_de\_personas ++}
   \State{UNLOCK(mutex\_cantidad\_de\_personas)}
   
   \State{LOCK(mutex\_posiciones[alumno.fila][alumno.columna])}
     \IndState{aula.posiciones[alumno.fila][alumno.columna] ++}
   \State{UNLOCK(mutex\_posiciones[alumno.fila][alumno.columna])}
 \EndProcedure
 \bigskip
 
 
 \Procedure{t\_aula\_liberar}{$aula,\ alumno$}
   \State{LOCK(mutex\_cantidad\_de\_personas)}
     \IndState{aula.cantidad\_de\_personas - -}
   \State{UNLOCK(mutex\_cantidad\_de\_personas)}
 \EndProcedure
 \bigskip
 
 
 \Procedure{terminar\_servidor\_de\_alumno}{socket\_fd, aula, alumno}
   \State{close(socket\_fd}
   \State{t\_aula\_liberar(aula, alumno)}
   \State{exit(-1)}
 \EndProcedure
 \bigskip
 
 
 \Function{intentar\_moverse}{aula, alumno, direccion}
   \State{Calculo la nueva\_posicion = (fila, columna)}
   
   \If{nueva\_posicion es la salida}
     \State{alumno.salio = true}
     \State{pudo\_moverse = true}
   \EndIf
   
   \If{nueva\_posicion esta entre límites \textbf{\&\&} aula.posiciones[nueva\_posicion] < Máx.cant. personas}
     \State{pudo\_moverse = true}
   \EndIf
   \State{vieja\_posicion = (alumno.fila, alumno.columna)}
   
   \If{nueva\_posicion.fila < vieja\_posicion.fila \textbf{||} (nueva\_posicion.fila == vieja\_posicion.fila \textbf{\&\&} nueva\_posicion.columna < vieja\_posicion.columna)}
     \If{!alumno.salio}
       \State{LOCK(mutex\_posiciones[nueva\_posicion.fila][nueva\_posicion.columna])}
     \EndIf
     \State{LOCK(mutex\_posiciones[vieja\_posicion.fila][vieja\_posicion.columna])}
     \State{orden\_locks = 0}
   \Else
     \State{LOCK(mutex\_posiciones[vieja\_posicion.fila][vieja\_posicion.columna])}
     \If{!alumno.salio}
       \State{LOCK(mutex\_posiciones[nueva\_posicion.fila][nueva\_posicion.columna])}
     \EndIf
     \State{orden\_locks = 1}
   \EndIf
   
   \If{pudo\_moverse}
     \If{!alumno.salio}
       \State{aula.posiciones[nueva\_posicion] ++}
     \EndIf
     \State{aula.posiciones[vieja\_posicion] - -}
     \State{actualizar posicion del alumno}
   \EndIf
   
   \If{orden\_locks == 0}
     \State{UNLOCK(mutex\_posiciones[vieja\_posicion.fila][vieja\_posicion.columna])}
     \If{!alumno.salio}
       \State{UNLOCK(mutex\_posiciones[nueva\_posicion.fila][nueva\_posicion.columna])}
     \EndIf
   \Else
     \If{!alumno.salio}
       \State{UNLOCK(mutex\_posiciones[nueva\_posicion.fila][nueva\_posicion.columna])}
     \EndIf
     \State{UNLOCK(mutex\_posiciones[vieja\_posicion.fila][vieja\_posicion.columna])}
   \EndIf
   \State{\textbf{return} pudo\_moverse}
 \EndFunction
 \bigskip
 
 
 \Procedure{colocar\_mascara}{$aula,\ alumno$}
   \State{alumno.tiene\_mascara = true}
 \EndProcedure
 \bigskip
 
 
 \Procedure{atendedor\_de\_alumno}{$socket\_fd,\ aula,\ salidas,\ alumno$}
   \If{no se pudo recibir el nombre y la posicion}
     \State{terminar\_servidor\_de\_alumno(socket\_fd, NULL, NULL)}
   \EndIf
   \State{t\_aula\_ingresar(aula, alumno)}
   
   \For{Ever}
     \If{no se pudo recibir la direccion}
       \State{terminar\_servidor\_de\_alumno(socket\_fd, aula, alumno)}
     \EndIf
     \State{pudo\_moverse = intentar\_moverse(aula, alumno, direccion)}
     \State{enviar\_respuesta(socket\_fd, pudo\_moverse)}     
     \If{alumno.salio}
       \State{break}
     \EndIf
   \EndFor
   
   \State{LOCK(mutex\_rescatistas)}
      \While{aula.rescatistas\_disponibles == 0}
	 \State{COND\_WAIT(cond\_hay\_rescatistas, mutex\_rescatistas)}
      \EndWhile
      
      \IndState{aula.rescatistas\_disponibles - -}
   \State{UNLOCK(mutex\_rescatistas)}
   
   \State{colocar\_mascara(aula, alumno)}
   
   \State{LOCK(mutex\_rescatistas)}
     \State{aula.rescatistas\_disponibles ++}
     \State{COND\_SIGNAL(cond\_hay\_rescatistas)}
   \State{UNLOCK(mutex\_rescatistas)}
   
   \State{LOCK(mutex\_pasillo)}
     \State{t\_aula\_liberar(aula, alumno)}
     \State{salidas.cant\_personas\_pasillo ++}
   \State{UNLOCK(mutex\_pasillo)}
   
   \State{LOCK(mutex\_grupo)}
     \While{salidas.cant\_personas\_grupo >= 5}
	 \State{COND\_WAIT(cond\_grupo\_lleno, mutex\_grupo)}
     \EndWhile
     
     \State{LOCK(mutex\_pasillo)}
       \State{salidas.cant\_personas\_pasillo - -}
       \State{salidas.cant\_personas\_grupo ++}
     \State{UNLOCK(mutex\_pasillo)}
     
     \If{salidas.cant\_personas\_grupo == 5 \textbf{||} (salidas.cant\_personas\_pasillo == 0 \textbf{\&\&} aula.cantidad\_de\_personas == 0)}
       \State{COND\_BROADCAST(cond\_estan\_evacuando)}
     \Else
       \State{COND\_WAIT(cond\_estan\_evacuando, mutex\_grupo)}
     \EndIf
     
     \State{salidas.cant\_personas\_grupo - -}
     \If{salidas.cant\_personas\_grupo > 0}
       \State{\State{COND\_WAIT(cond\_salimos\_todos, mutex\_grupo)}}
     \Else
       \State{COND\_BROADCAST(cond\_salimos\_todos)}
       \State{COND\_BROADCAST(cond\_grupo\_lleno)}
     \EndIf
   \State{UNLOCK(mutex\_grupo)}
   
   \State{enviar\_respuesta(socket\_fd, LIBRE)}
 \EndProcedure
 \bigskip
 
 
 \Function{main}{void}
   \State{Inicializar conexion con sockets y permitir 5 conexiones en espera}
   \State{t\_aula\_iniciar\_vacia(aula)}
   \State{Inicializar mutexes}
   \State{Inicializar variables de condicion}
   \For{Ever}\Comment{Loop de atencion al cliente}
     \If{Se conecto un nuevo cliente}
       \State{Inicializar pthread y sus parametros}
       \State{Crear pthread con la funcion 'atendedor\_de\_alumno'}
     \EndIf
   \EndFor
 \EndFunction
 
\end{algorithmic}

%TODO: la explicación de cada función, explicación del pasillo y como se retiran los grupos.

\subsection{Deadlock}
Un grupo de procesos están en estado de $deadlock$ si cada uno de ellos está esperando un evento que sólo otro proceso del grupo puede causar.
Vamos a analizar el código para demostrar que está libre de deadlocks.

Antes de eso veremos cuáles son las condiciones que debe cumplir un sistema para tener la posibilidad de llegar a un estado de deadlock, llamadas
condiciones de Coffman:
\begin{itemize}
 \item Exclusión mutua: cada recurso está asignado exactamente a un proceso o está disponible.
 \item Hold-and-wait: Los procesos que tienen asignado un recurso pueden requerir otro/s recurso/s.
 \item No-preemption: Los recursos asignados a procesos no pueden ser removidos por la fuerza.
 \item Espera circular: Debe haber una lista de dos o más procesos, cada uno de ellos esperando un recurso que tiene el anterior.
\end{itemize}

En nuestra implementación la condición de Coffman que no se cumple es la última, por ende está libre de deadlocks.
Entonces queremos probar que no hay espera circular:

Para asegurarnos de que no haya espera circular en cuanto a las posiciones del aula lo que hicimos fue disponer un orden de bloqueos.
%TODO: agregar gráfico o explicación del orden
De esta forma, si dos clientes deben bloquear las posiciones 1 y 2. Ambos lo van a hacer en el mismo orden. Entonces uno de ellos... %TODO


%....................TODO:

No hay espera circular porque imponemos un orden a como tomar los locks de las posiciones del aula. 
Agregar grafiquito de la matriz con sus flechitas de mayor, menor o casos.

Hold and wait sin espera circular: nadie que tenga el pasillo va a pedir el grupo entonces siempre se pide el grupo en algún momento se libera el pasillo.
Nunca se da la situación de que alguien tiene el grupo y esta esperando el pasillo mientras otro tiene el pasillo mientras espera el grupo.
Orden determinado, primero grupo, dsp pasillo, dsp aula ponele.
Más grafiquitos, los está haciendo Dami

Una variante de deadlock es livelock, algo de explicación, no pasa porq asumimos q los clientes son personas lógicas!

\subsection{Paralelismo}
El objetivo de esta implementación es maximizar el grado de paralelismo, por esta razón decidimos utilizar semáforos independientes para las posiciones
del aula y sus atributos.

En un principio nuestra solución utilizaba un unico mutex para restringir el acceso a la estructura del aula, si bien esta solución funcionaba
no se permitian movimientos paralelos dentro del aula, aunque esas posiciones fueran totalmente distintas, lo cual quita paralelismo.

Además se buscó reducir las secciones críticas lo mayor posible. Así, por ejemplo, mientras a un alumno le están poniendo una máscara otro puede bloquear el mutex
de los rescatistas ya que no está siendo modificada la cantidad de ellos.

