%Justificar en todos los casos todas las decisiones tomadas.
%Implementación realizada: indicando los archivos creados y/o modificados. Dificultades encontradas
%Qué tipos de datos usamos para sincronizar los procesos. Por qué
\subsection{Variables utilizadas}
\textbf{Compartidas:} t\_aula el\_aula. Es una estructura que almacena los datos de una reserva.\\
Contiene: 
\begin{itemize}
 \item Una matríz de enteros donde cada posición equivale a $1m^{2}$ del aula.
 \item Cantidad de personas dentro del aula.
 \item Cantidad de rescatístas.
\end{itemize}

\textbf{Privadas:} t\_persona alumno. 
Contiene:
\begin{itemize}
 \item Nombre
 \item Posición en la que se encuentra (fila y columna)
 \item Un booleano indicando si salió y otro si tiene la máscara.
\end{itemize}

\textbf{Caso especial:} t\_parametros. Es una estructura que contiene lo siguiente:
\begin{itemize}
 \item Identificador del socket que utilizará el pthread para comunicarse con el cliente.
 Este socket es único para cada pthread.
 \item Variable compartida el\_aula.
\end{itemize}

Es necesario tener una estructura para pasar los parámetros porque la implementación de pthreads sólo admite el pasaje de un parámetro y debe
ser una estructura.\\
Al crearlo utilizamos malloc para alojarlo en memoria, de no hacerlo de esta forma la variable \textquoteleft muere\textquoteright al final del $scope$ y
el pthread estaría accediendo a posiciones inválidas (contienen basura).\\

\subsection{Mutexes}
%pthread_mutex_t mutex_aula;			explicar cuándo se usa c/u y por qué no hay deadlock (básicamente no hay hold & wait)
%pthread_mutex_t mutex_rescatistas;
%pthread_cond_t cond_hay_rescatistas;

\subsection{Deadlock}

\subsection{Implementación}
%servidor_multi y un algorithm explicando el atendedor de boludos eee digo alumnos
