\subsubsection{Compartidas}
\textbf{t\_aula aula.}
Estructura que contiene: 
\begin{itemize}
 \item Una matríz de enteros donde cada posición equivale a $1\ m^{2}$ del aula.
 \item Cantidad de personas dentro del aula.
 \item Cantidad de rescatístas.
\end{itemize}

\subsubsection{Privadas}
\textbf{t\_persona alumno.}
Estructura que contiene: 
\begin{itemize}
 \item Nombre
 \item Posición en la que se encuentra (fila y columna)
 \item Un booleano indicando si salió y otro si tiene la máscara.
\end{itemize}

\smallskip
\textbf{t\_salidas salidas.}
Estructura que contiene: 
\begin{itemize}
 \item Cantidad de personas en el pasillo. 
 El pasillo es una zona intermedia entre el aula y el grupo. En él se encuentran 
 los alumnos con máscara esperando para poder ingresar en un grupo.
 \item Cantidad de personas en el grupo. 
 En el grupo los alumnos esperan ser cinco para poder ser evacuados.
\end{itemize}

\subsubsection{Caso especial}
\textbf{t\_parametros.} 
Estructura que contiene: 
\begin{itemize}
 \item Identificador del socket que utilizará el pthread para comunicarse con el cliente.
 Este socket es único para cada pthread.
 \item Variable compartida $el\_ aula$.
 \item Variable privada $salidas$.
\end{itemize}

La implementación de pthreads sólo admite el pasaje de un parámetro y debe
ser una estructura, por eso t\_parametros es necesaria. Al crearla utilizamos malloc para alojarla en memoria, 
de no hacerlo de esta forma la variable \textquoteleft muere\textquoteright al final del $scope$ y
el pthread estaría accediendo a posiciones inválidas (contienen basura).