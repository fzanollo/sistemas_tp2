En los sistemas operativos tradicionales, cada proceso tiene un espacio de dirección y un único hilo (o thread) de control. De hecho, esta es casi
la definición de proceso. Sin embargo, en muchas situaciones, es deseable tener múltiples threads de control en el mismo espacio de dirección corriendo 
quasi-paralelamente, como si fueran (casi) procesos diferentes (excepto por el espacio de direcciones compartido).

La razón principal para utilizar estos microprocesos, llamados threads\cite{Tanen}, es que en muchas aplicaciones hay múltiples actividades ocurriendo al mismo tiempo.
Algunas de ellas pueden bloquearse de vez en cuando. Al descomponer tal aplicación en múltiples threads secuenciales que corren quasi-paralelamente, el 
modelo de programación se vuelve más simple.

Un segundo argumento de por qué tenerlos es que, al ser más livianos que los procesos, es más fácil (y por lo tanto más rápido) crearlos y destruirlos.

La tercer razón para usarlos tiene que ver con rendimiento. Los threads no mejoran el rendimiento si todos ellos están ligados a un CPU, pero cuando
hay una gran cantidad de CPU y dispositivos de I/O, tener threads permite a las actividades superponerse entre sí, mejorando así el rendimiento.

Por último, los threads son útiles en sistemas con múltiples CPUs, dónde es posible el paralelismo real.

Históricamente, los proveedores de hardware implementaban su propia versión de threads. Estas implementaciones diferían sustancialmente entre sí, haciendo difícil
para los programadores desarrollar aplicaciones portables que utilicen threads.\cite{Pthreads}

Para hacer posible la portabilidad de programas, la IEEE definió un estándar. Las implementaciones que adhieren a este estándar son referidas como
POSIX threads o \textbf{Pthreads}. Los Pthreads están definidos como una librería del lenguaje C, conteniendo de tipos de datos y llamadas a procedimientos.
La mayoría de los sistemas UNIX la soportan. El estándar define más de 60 funciones.\\
Cada pthread tiene las siguientes propiedades:
\begin{itemize}
 \item Identificador.
 \item Set de registros (incluido el $program\ counter$).
 \item Set de atributos (que son guardados en una estructura e incluyen el tamaño del $stack$ entre otros).\\
\end{itemize}

Pthreads también proporciona un número de funciones que pueden ser usadas para sincronizar threads. El mecanismo más básico usa una variable
de exclusión mutua o \textbf{mutex}, que puede ser bloqueada o desbloqueada para proteger una sección crítica.\\
Si un thread quiere entrar en una sección crítica primero trata de bloquear el mutex asociado a esa sección. Si estaba desbloqueado entra y 
el mutex es bloqueado automáticamente, previniendo así que otros threads entren. Si ya estaba bloqueado, el thread se bloquea hasta que el
mutex sea liberado.\\ 
Si múltiples threads están esperando el mismo mutex, cuando éste es desbloqueado, sólo le es permitido continuar a uno de ellos.
Estos bloqueos no son mandatorios. Queda a cargo del programador asegurar que los threads los utilizan correctamente.\\

Además de los mutex, Pthreads ofrece un segundo mecanismo de sincronización: \textbf{variables de condición}. Éstas permiten a los threads 
bloquearse debido al incumplimiento de alguna condición. Casi siempre estos mecanismos son utilizados conjuntamente.

%TODO agregar a el señor tanembaun a la biblio y https://computing.llnl.gov/tutorials/pthreads/#Abstract (leer más arriba, lo use para dos o tres párrafos)