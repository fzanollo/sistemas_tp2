Un grupo de procesos están en estado de $deadlock$ si cada uno de ellos está esperando un evento que sólo otro proceso del grupo puede causar.
Vamos a analizar el código para demostrar que está libre de deadlocks.

Antes de eso veremos cuáles son las condiciones que debe cumplir un sistema para tener la posibilidad de llegar a un estado de deadlock, llamadas
condiciones de Coffman:
\begin{itemize}
 \item Exclusión mutua: cada recurso está asignado exactamente a un proceso o está disponible.
 \item Hold-and-wait: Los procesos que tienen asignado un recurso pueden requerir otro/s recurso/s.
 \item No-preemption: Los recursos asignados a procesos no pueden ser removidos por la fuerza.
 \item Espera circular: Debe haber una lista de dos o más procesos, cada uno de ellos esperando un recurso que tiene el anterior.
\end{itemize}

En nuestra implementación la condición de Coffman que no se cumple es la última, por ende está libre de deadlocks.
Entonces queremos probar que no hay espera circular:

Para asegurarnos de que no haya espera circular en cuanto a las posiciones del aula lo que hicimos fue disponer un orden de bloqueos.
%TODO: agregar gráfico o explicación del orden
De esta forma, si dos clientes deben bloquear las posiciones 1 y 2. Ambos lo van a hacer en el mismo orden. Entonces uno de ellos... %TODO


%....................TODO:

No hay espera circular porque imponemos un orden a como tomar los locks de las posiciones del aula. 
Agregar grafiquito de la matriz con sus flechitas de mayor, menor o casos.

Hold and wait sin espera circular: nadie que tenga el pasillo va a pedir el grupo entonces siempre se pide el grupo en algún momento se libera el pasillo.
Nunca se da la situación de que alguien tiene el grupo y esta esperando el pasillo mientras otro tiene el pasillo mientras espera el grupo.
Orden determinado, primero grupo, dsp pasillo, dsp aula ponele.
Más grafiquitos, los está haciendo Dami

Una variante de deadlock es livelock, algo de explicación, no pasa porq asumimos q los clientes son personas lógicas!