\label{sec:deadlock}
Un grupo de procesos están en estado de $deadlock$ si cada uno de ellos está esperando un evento que sólo otro proceso del grupo puede causar.
Vamos a analizar el código para demostrar que está libre de deadlocks.

Antes de eso veremos cuáles son las condiciones que debe cumplir un sistema para tener la posibilidad de llegar a un estado de deadlock, llamadas
condiciones de Coffman:
\begin{itemize}
 \item Exclusión mutua: cada recurso está asignado exactamente a un proceso o está disponible.
 \item Hold-and-wait: Los procesos que tienen asignado un recurso pueden requerir otro/s recurso/s.
 \item No-preemption: Los recursos asignados a procesos no pueden ser removidos por la fuerza.
 \item Espera circular: Debe haber una lista de dos o más procesos, cada uno de ellos esperando un recurso que tiene el anterior.
\end{itemize}

En nuestra implementación no se cumple la última condición de Coffman, por ende está libre de deadlocks.
Entonces veamos por qué no hay espera circular:

Para asegurarnos de que no haya espera circular en cuanto a las posiciones del aula lo que hicimos fue disponer un orden de bloqueos.
%TODO: agregar gráfico o explicación del orden
De esta forma, si dos clientes deben bloquear las posiciones 1 y 2. Ambos lo van a hacer en el mismo orden. Entonces si 1 < 2, uno de ellos tomará el lock
de la posición 1 y el otro se quedará esperando a que se libere. Núnca sucede que uno de ellos toma el 2 sin tener el 1.

Otro momento donde hay hold and wait es cuando el cliente tiene el mutex del grupo y pide el del pasillo. Pero no hay ningún lugar en el código donde 
pueda llegar a pedir primero el pasillo y luego el grupo. Notar que sí hay un pedido del mutex del pasillo sin el del grupo, pero en ese caso
el cliente eventualmente desbloqueará el pasillo, permitiendo a los demás seguir su curso.
%....................TODO:
Grupo-pasillo-aula, explicar porq no hay deadlock con eso
Más grafiquitos, los está haciendo Dami
%....................

Una variante de deadlock es livelock, ambos son formas de inanición. Excepto que el estado de los procesos envueltos en el livelock está cambiando constantemente,
aunque sin avanzar. Al suponer que los clientes son personas lógicas evitamos caer en un livelock, ya que no tiene sentido moverse en círculos dentro del aula
en vez de salir. %TODO: esto está horriblemente horrible, aprendé a redactar!